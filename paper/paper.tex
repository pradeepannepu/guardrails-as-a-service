\documentclass[conference]{IEEEtran}
\usepackage{graphicx}
\usepackage{listings}
\usepackage{url}
\usepackage{hyperref}
\usepackage{amsmath}
\usepackage{enumitem}
\hypersetup{colorlinks=true,linkcolor=black,citecolor=black,urlcolor=blue}

\title{Guardrails-as-a-Service}

\author{
\IEEEauthorblockN{Pradeep Annepu, Akula Rajesh, Amarendra Chakravarthi, Anirudh Reddy}
\IEEEauthorblockA{Email: \{M25AI1109, M25AI1048, M25AI1082, M25AI1131\}@iitj.ac.in}
}

\begin{document}
\maketitle

\begin{abstract}
  We present the architecture and planned evaluation of a Guardrails-as-a-Service platform. The system is designed to scale: it runs as stateless microservices connected by asynchronous events.It’s easy to extend because new handlers can be plugged in without changing the core. Every decision was logged in event driven way to scale independently. We trianed a gemma model with custom domain language to act as a guard model. We define measurable quality attributes (latency, throughput) using a load testing.
\end{abstract}

\begin{IEEEkeywords}
  Policy enforcement, microservices, vector similarity, audit logging, observability, gemma, guardrails.
\end{IEEEkeywords}

\section{Introduction}
Enterprises increasingly rely on the agentic systems and exposing varous tools and capabilities to their customers. The need for guarding the tools and capabilities is also increasing with these new systems. Training a large language model from the scratch can be a expensive affair for the many enterprises and many make use of the open source/weight llm models available in the market. These open models have trained in general and may not be alighed with enterprises needs and their domain knowledge. Hence the need for guardrails to these models is required for enterpises to avoid any legal and compliance issues. It is also very important to save their inference cost by avoiding any unwanted/unnecessary calls to the llm models. For example, if an enterprise of payment gateway is exposing an agentic systems to their customers to help with their support queries, but hackers may try to exploit the system to get the sensitive information or even play with it general discussion which may not be related to the customer business, which can lead to huge inference cost to the company.

It is not a one time job for these enterprises to set these guardrails and forget about it, as the business needs and complaince requirements keep changing, these guardrails need to be updated frequently. Hence the need for a policy service which can help enterprises to create, update and manage these guardrails easily.We address these gaps by designing a scalable system for the guardrails evaluation and continuous policy updates and management in this study.

\section{Contributions}
The work provides: (1) A evaluation architecture combining deterministic rule filtering of plicies for guards; (2) An extensible handler interface enabling emerging policy types with low change cost; (3) A audit log for tamper-evidence; (4) A metrics-driven evaluation plan for continous eval and agentic system that use llms.

\section{System Architecture}
\begin{figure}[htbp]
  \centerline{\includegraphics[width=\columnwidth]{services.png}}
  \caption{System Architecture}
  \label{fig:services}
\end{figure}

Refer Fig.~\ref{fig:services} For System Architecture

\noindent Core components:
\begin{itemize}[leftmargin=*]
  \item API Gateway: Authentication, authorization, rate limiting.
  \item Policy Service: CRUD, versioning, vector indexing (pgvector \cite{pgvector}), and retrieval.
  \item Policy Handlers: Rule-based evaluation, semantic filtering, custom models and custom handlers.
  \item Evaluation Plane: Cache-first retrieval, rule + semantic evaluation, decision event emission.
  \item Audit Logger: Consumes decision events; persists hash-chained records.
  \item Redis: Low-latency hot-set and embedding cache.
  \item Broker: Decouples evaluation from persistence; supports resilience tests.
  \item Observability Stack: Metrics (Prometheus \cite{prometheus}), logs, planned traces (OpenTelemetry \cite{otel}).
\end{itemize}

\section{Design Methodology}
Microservice decomposition isolates policy mutation from evaluation throughput. Event-driven processing reduces synchronous coupling. Vector similarity augments metadata filtering, allowing semantically proximate policy selection. Hash chaining supplies integrity assurance with linear verification. Refer to Fig.~\ref{fig:c4} for the system context diagram.

\begin{figure*}[htbp]
  \centerline{\includegraphics[width=\textwidth]{c4.png}}
  \caption{C4 - System context diagram}
  \label{fig:c4}
\end{figure*}

\subsection{Custom Guard Model}
We have trained the Gemma model \cite{gemma} with custom domain language to act as a guard model. We choosen the networking, interconnections and datacenter terminologies as allowed and disallowed to respond with any general knowledge.

Custom trained model is inferenced via micro service (inference service via python fast api). The policy service was dynamic and can accept muliple policy handlers, Our custom model was one of the handler
\subsection{Domain Model}
Refer to Fig.~\ref{fig:domain_model} for Domain model, entities, we made it customizable to accept new policy types and handlers and choosen SQL based relational model for easy extensibility. \\
\textbf{EvaluationRequest}: A user's request to evaluate one or more model predictions against a set of policies and rules. \\
\textbf{EvaluationResult}: The outcome of processing an EvaluationRequest (including which rules were violated). \\
\textbf{Policy}: A compliance, safety, quality, or business policy composed of one or more rules. \\
\textbf{Rule}: A specific, testable condition inside a Policy. \\
Violation: A record that a particular Rule (within its Policy) was broken during evaluation. \\
\textbf{ModelPrediction}: An individual output (e.g., a generated text, classification, score) produced by a model that is being evaluated. \\
\textbf{AuditLog}: A formal log entry capturing the evaluation event for traceability and governance. \\
\textbf{User}: The actor (human or system identity) who initiated the evaluation.

\begin{figure}[htbp]
  \centerline{\includegraphics[width=\columnwidth]{domain_model.png}}
  \caption{Domain Model: Policy, Handler, Decision Event}
  \label{fig:domain_model}
\end{figure}

\subsection{Sequence Flow}
Refer to Fig.~\ref{fig:sequence} for Sequence diagram of policy evaluation flow.
\begin{itemize}
  \item Client request with context hits API gateway. Authentication, authorization, rate limiting are performed in this layer.
  \item Policy Service retrieves candidate policies via metadata filtering + vector similarity
  \item Evaluation Plane invokes appropriate Policy Handlers. Our Custom model handler is one of them. which is trained to allow only networking and intercoonnections related queries.
  \item Handlers evaluate context against policies; emit Decision Events.
  \item Audit Logger persists hash-linked Decision Events for continous evaluation of the system.
\end{itemize}


\begin{figure*}[htbp]
  \centerline{\includegraphics[width=\textwidth]{sequence.png}}
  \caption{Sequence Diagram: Policy Evaluation Flow}
  \label{fig:sequence}
\end{figure*}

\section{Quality Attributes}
\subsection{Scalability}
Tactics: Stateless pods, horizontal autoscaling, Redis caching, asynchronous buffering. Metrics: P95 latency $<150$\,ms under 2000 requests/sec; cache hit ratio correlated with latency reduction.

\begin{figure*}[htbp]
  \centerline{\includegraphics[width=\textwidth]{load_test_result_1.png}}
  \caption{Load test result by k6}
  \label{fig:load_test_result_1}
\end{figure*}

\subsection{Extensibility}
A \texttt{PolicyHandler} interface:
\begin{lstlisting}[language=java,basicstyle=\ttfamily\small]
export interface PolicyHandler {
  supports(policyType: string): boolean;
  evaluate(context: any, policy: Policy): Promise;
}
\end{lstlisting}
New handlers (e.g., advanced semantic constraint) integrate via dependency injection without core modification. Verification: Diff locality; unchanged regression suite; cyclomatic complexity < 10.

\subsection{Observability \& Auditability}
Correlation IDs propagate through logs and metrics. Audit integrity uses chained hashes:
\begin{lstlisting}[language=Python,basicstyle=\ttfamily\small]
prev_hash = "GENESIS"
for event in decision_events:
    serialized = canonical_json(event)
    curr_hash = sha256(prev_hash + serialized)
    store({...event, prev_hash, curr_hash})
    prev_hash = curr_hash
\end{lstlisting}
Periodic verification recomputes sequence and alerts on mismatch; target audit lag < 5\,s at sustained load.

\subsection{Maintainability}
Modular boundaries and standardized evaluation signatures reduce change effort. KPIs: Mean Time To Change for routine policy extension < 0.5 developer-day.

\section{Evaluation Plan}
Phases: (1) Load tests (k6 ramp to 2000 req/sec); (2) Resilience (broker partition throttling). Metrics scraped every 5\,s; Grafana dashboards visualize throughput, latency, audit lag. Success: No message loss; hash chain validates end-to-end; latency targets met.

We ran a load test using k6 tool and the results are shown in Fig.~\ref{fig:load_test_result_1}.

\section{Security Considerations}
Current: JWT-based client auth; least-privilege DB roles; rate limiting. Planned: mTLS inter-service; policy signature verification for provenance.

\section{Related Work}
OPA \cite{opa} offers strong declarative evaluation but lacks semantic retrieval integration and native hash-chained audit logging. AWS Control Tower guardrails \cite{awsguardrails} provide opinionated cloud governance with limited cross-cloud extensibility. Our approach differentiates through hybrid retrieval, event-driven decoupling, and cryptographic audit integrity.

\section{Conclusion}
The proposed Guardrails-as-a-Service platform aligns architectural tactics with measurable quality attributes. Hybrid policy evaluation and hash-linked auditing together enhance governance fidelity and trust. Future work targets production hardening, expanded semantic handlers, and full distributed tracing.

\section*{Acknowledgment}
we would like to thank the Indian Institute of Technology Jodhpur for supporting this study.

\begin{thebibliography}{99}
  \bibitem{opa} Open Policy Agent Documentation. [Online]. Available: \url{https://www.openpolicyagent.org}
  \bibitem{awsguardrails} AWS Control Tower Guardrails. [Online]. Available: \url{https://docs.aws.amazon.com/controltower}
  \bibitem{prometheus} Prometheus Documentation. [Online]. Available: \url{https://prometheus.io}
  \bibitem{pgvector} pgvector Extension Documentation. [Online]. Available: \url{https://github.com/pgvector/pgvector}
  \bibitem{otel} OpenTelemetry Specification. [Online]. Available: \url{https://opentelemetry.io}
  \bibitem{gemma} Gemma 3 Technical Report \url{https://arxiv.org/pdf/2503.19786}
  \bibitem{DuoGuard} (2025) DuoGuard: A Two-Player RL-Driven Framework for Multilingual LLM Guardrails. arXiv preprint arXiv:2502.05163. Available: \url{https://arxiv.org/abs/2502.05163}
  \bibitem{rebedea-etal-2023-nemo} {N}e{M}o Guardrails: A Toolkit for Controllable and Safe {LLM} Applications with Programmable Rails \url{https://aclanthology.org/2023.emnlp-demo.40/}
\end{thebibliography}

\end{document}
